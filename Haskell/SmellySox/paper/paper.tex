\documentclass{llncs}

\usepackage{amsmath}
\usepackage{algorithm}
\usepackage{algpseudocode} 

\title{Sort it out with Monotonicity}
\author{Ann Lilliestr\"om and Nick Smallbone}
\institute{ Chalmers University of Technology, Gothenburg, Sweden
         \\ \email{\{annl,nicsma\}@chalmers.se}
          }

\begin{document}

\maketitle

\begin{abstract}
None of the famous theorem provers for first-order logic deal with
many-sorted problems. Ideally, we would like our provers to understand
many-sorted logic, but in the meantime, we want to be able to
\emph{translate} a sorted problem into an unsorted one so that the
usual crowd of theorem provers and model finders can solve it.

There are two obvious ways to do this. One is a) easy, b) efficient
and c) often absolutely wrong, while the other is a) always correct
and b) extremely wasteful. We present a technique called
\emph{monotonicity inference} that allows us to use a mixture of the
two techniques: the efficient one when sound and the wasteful one when
necessary. To demonstrate this technique we wrote a new tool,
Monopoly, the first of its kind as far as we know.
\end{abstract}


\section{Introduction}

Some cool example (puzzle? monopoly?)

Monotonicity for FOL, sort inference, "sort erasure"





\section{Sort Erasure}

One obvious approach is to simply erase all the sorts: that is,
whenever we have a sorted quantification $(\forall x:S)\ldots$ or
$(\exists x:S)\ldots$, we just turn it into an unsorted quantification
$(\forall x)\ldots$ or $(\exists x)\ldots$.

Unfortunately, this doesn't work. Consider the formula
\begin{align*}
& (\forall x y:A)(x = y) \\
\land & (\exists x y:B)(x \neq y)
\end{align*}
This formula is satisfiable: for a model, let the domain of $A$ have
size 1 and the domain of $B$ have size greater than 1. If we erase the
sorts, though, we get $(\forall x y)(x = y) \land (\exists x y)(x \neq
y)$, which is of the form $P \land \neg P$ and therefore
unsatisfiable.

\subsection{Introducing sorting predicates}

... For each sort $S$, we introduce a predicate $p_S$, of arity one,
so that $p_S(x)$ means ``$x$ has sort $S$''. A
$\forall$-quantification $(\forall x:S)\ldots$ becomes $(\forall
x)(p_S(x) \Rightarrow \ldots)$ and an $\exists$-quantification
$(\exists x:S)\ldots$ becomes $(\exists x)(p_S(x) \land \ldots)$. For
each function symbol we generate an axiom giving its sort; for
example, $f : (S \times T \Rightarrow U)$ induces an axiom
\begin{displaymath}
(\forall x y)(p_S(x) \land p_T(y) \Rightarrow p_U(f(x, y)))
\end{displaymath}

...

This translation is sound, but inefficient: we give the theorem prover
a lot of extra work to do in proving that things are of the right
sort. Whenever we instantiate an axiom of the form $(\forall
x:S)P(x)$ to get $P(t)$ we get an extra proof obligation $p_S(t)$.

\subsection{Introducing sorting functions}

...

\subsection{Introducing stuff only when necessary}

...

\section{Monotonicity calculus for first-order logic}


Definition:
A formula f is monotone in a type A, if

  If f is satisfiable in a model where the domain of A has size n,
  then it is also satisfiable for any domain size $>$ n.
  (Unsatisfiable formulae are always monotone).

A formula is anti-monotone in a type A if:

  If f is unsatisfiable in a model where the domain of A has size n,
  then it is also unsatisfiable for any domain size $>$ n.

Monotonicity inference is semi-decidable:

Reduce solving diophantine equations to monotonicity

To show that a problem is not monotone, we look for a domain size
k for which the problem is satisfiable, and not satisfiable for size k+1.
This can be done in finite time using a finite model finder.



\subsection{Simple calculus (extend only by copy)}

- example where doesn't work

\subsection{Better calculus (extend by true/false/copy)}


\subsection{ "Biggest" sort, finding injective functions, infinite sorts...}

-Perhaps move to Future work.






















\documentclass{report}

\usepackage{amsmath}

\title{Sort it out with monotonicity}
\author{Ann Lilliestr\"om and Nick Smallbone}

\begin{document}

\maketitle

\begin{abstract}
None of the famous theorem provers for first-order logic deal with
many-sorted problems. Ideally, we would like our provers to understand
many-sorted logic, but in the meantime, we want to be able to
\emph{translate} a sorted problem into an unsorted one so that the
usual crowd of theorem provers and model finders can solve it.

There are two obvious ways to do this. One is a) easy, b) efficient
and c) often absolutely wrong, while the other is a) always correct
and b) extremely wasteful. We present a technique called
\emph{monotonicity inference} that allows us to use a mixture of the
two techniques: the efficient one when sound and the wasteful one when
necessary. To demonstrate this technique we wrote a new tool,
Monopoly, the first of its kind as far as we know.
\end{abstract}

\subsection{Erasing all the sorts}

One obvious approach is to simply erase all the sorts: that is,
whenever we have a typed quantification $(\forall x:S)\ldots$ or
$(\exists x:S)\ldots$, we just turn it into an untyped quantification
$(\forall x)\ldots$ or $(\exists x)\ldots$.

Unfortunately, this doesn't work. Consider the formula
\begin{align*}
& (\forall x y:A)(x = y) \\
\land & (\exists x y:B)(x \neq y)
\end{align*}
This formula is satisfiable: for a model, let the domain of $A$ have
size 1 and the domain of $B$ have size greater than 1. If we erase the
types, though, we get $(\forall x y)(x = y) \land (\exists x y)(x \neq
y)$, which is of the form $P \land \neg P$ and therefore
unsatisfiable.

\subsection{Introducing sorting predicates}

... For each sort $S$, we introduce a predicate $p_S$, of arity one,
so that $p_S(x)$ means ``$x$ has sort $S$''. A
$\forall$-quantification $(\forall x:S)\ldots$ becomes $(\forall
x)(p_S(x) \Rightarrow \ldots)$ and an $\exists$-quantification
$(\exists x:S)\ldots$ becomes $(\exists x)(p_S(x) \land \ldots)$. For
each function symbol we generate an axiom giving its type; for
example, $f : (S \times T \Rightarrow U)$ induces an axiom
\begin{displaymath}
(\forall x y)(p_S(x) \land p_T(y) \Rightarrow p_U(f(x, y)))
\end{displaymath}

...

This translation is sound, but inefficient: we give the theorem prover
a lot of extra work to do in proving that things are of the right
type. Whenever we instantiate an axiom of the form $(\forall
x:S)P(x)$ to get $P(t)$ we get an extra proof obligation $p_S(t)$.

\end{document}

\section{Results}

compare simple/better calculus.

compare with and without searching for "biggest" sort
with injective functions



\section{Future Work}

* Sort inference
* Polymorphic predicates
* Improving the calculus (other types of predicate extensions)
* Other ways of representing sorts in unsorted FOL


\section{Conclusions}

We're awesome.
 


\end{document}
