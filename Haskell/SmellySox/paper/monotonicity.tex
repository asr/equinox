\section{Monotonicity calculus for first-order logic}


Definition:
A formula f is monotone in a type A, if

  If f is satisfiable in a model where the domain of A has size n,
  then it is also satisfiable for any domain size $>$ n.
  (Unsatisfiable formulae are always monotone).

A formula is anti-monotone in a type A if:

  If f is unsatisfiable in a model where the domain of A has size n,
  then it is also unsatisfiable for any domain size $>$ n.

Monotonicity inference is semi-decidable:

Reduce solving diophantine equations to monotonicity

To show that a problem is not monotone, we look for a domain size
k for which the problem is satisfiable, and not satisfiable for size k+1.
This can be done in finite time using a finite model finder.



\subsection{Simple calculus (extend only by copy)}

- example where doesn't work

\subsection{Better calculus (extend by true/false/copy)}


\subsection{ "Biggest" sort, finding injective functions, infinite sorts...}

-Perhaps move to Future work.





















