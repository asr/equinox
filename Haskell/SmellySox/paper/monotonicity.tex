\section{Monotonicity calculus for first-order logic}


Definition:
A formula f is monotonous in a type A, if

  If f is satisfiable in a model where the domain of A has size n,
  then it is also satisfiable for any domain size $>$ n.
  (Unsatisfiable formulae are always monotonous).

A formula is anti-monotonous in a type A if:

  If f is unsatisfiable in a model where the domain of A has size n,
  then it is also unsatisfiable for any domain size > n.

Monotonicity inference is semi-decidable:

To show that a problem is not monotone, we look for a domain size
k for which the problem is satisfiable, and not satisfiable for size k+1.
This can be done in finite time using a finite model finder.

For any closed HOL formula t, let

$$t' = t\ \vee \  \forall (X :: \alpha,Y :: \alpha)\ .\ x = y$$

where $\alpha$ does not occur in $t'$. 
If $t'$ is monotone in $\alpha$, then, since $t'$ is satisfiable for domain size 1, it must, by the definition of
monotonicity, be satisfiable for any domain size.
If $t'$ is not monotone in $\alpha$, then there is a domain size for which $t$ is not satisfiable.
Thus, we can reduce the problem of whether a formula is satisfiable in all domain sizes to a problem of monotonicity.
Since satisfiability in FOL is undecidable, monotonicity must be undecidable too.

\subsection{Simple calculus (extend only by copy)}

- example where doesn't work

\subsection{Better calculus (extend by true/false/copy)}


\subsection{ "Biggest" sort, finding injective functions, infinite sorts...}























