\section{Monotonicity calculus for first-order logic}

\label{sec_monotonicity}

In this section we treat monotonicity more formally and give a
calculus for inferring monotonicity of a formula. FIXME this is
rubbish

To begin with, we only consider monotonicity for unsorted formulae.
There is no real complication in the many-sorted case but the notation
is more cumbersome. Monotonicity is a semantic property rather than a
syntactic property of the formula.
\begin{definition}
An unsorted formula $\formula$ is \emph{monotone} if, for all $d$,
whenever $\formula$ is satisfiable over a domain of $d$ elements,
then $\formula$ is also satisfiable over a domain of $d+1$ elements.
\end{definition}

This definition only considers the formula's behaviour on
\emph{finite} domains---if $d$ is infinite, then $d+1 = d$ and the
condition is trivially satisfied. A trivial consequence is that if a
monotone formula is satisfiable for a domain of size $d$ then it is
satisfiable for all bigger domains.

FIXME write about satisfiability at a domain instead.

Several common classes of formulae are monotone:
\begin{itemize}
\item Any unsatisfiable formula is monotone because it has no models.
\item Any valid formula is monotone because it has a model no matter
  what the domain size.
\item A formula that only has infinite models is monotone because for
  all finite values of $d$ it has no models.
\item A formula that does not use $=$ is monotone, as we will see
  later.
\end{itemize}

The first two points suggest that monotonicity is related to
satisfiability and so should not be decidable. It is at least
semi-decidable: if a formula $\formula$ is not monotone then there
must be a domain size $d$ for which $\formula$ is satisfiable for
domains of size $d$ but not for domains of size $d+1$, and we can find
this $d$ using any model-finder. However, there is no algorithm that
can always tell us when a formula \emph{is} monotone; we leave the
proof to the appendix, but it is intuitively because monotonicity is
not easier than finite unsatisfiability.

What about a non-monotone formula? The simplest example is $(\forall
x, y)(x = y)$, which is satisfied if the domain contains a single
element but not if it contains two.

% A formula f is monotone in a type A, if

%   If f is satisfiable in a model where the domain of A has size n,
%   then it is also satisfiable for any domain size $>$ n.
%   (Unsatisfiable formulae are always monotone).

% A formula is anti-monotone in a type A if:

  % If f is unsatisfiable in a model where the domain of A has size n,
  % then it is also unsatisfiable for any domain size $>$ n.

Monotonicity inference is semi-decidable:

Reduce solving diophantine equations to monotonicity

To show that a problem is not monotone, we look for a domain size
k for which the problem is satisfiable, and not satisfiable for size k+1.
This can be done in finite time using a finite model finder.



In the many-sorted case we have one domain for each sort, which
complicates matters a little. We write $\M(\alpha)$ for the domain of
sort $\alpha$ in model $\M$.

\begin{definition}
A formula $\varphi$ is \emph{monotone} in a sort $\alpha$ if,
whenever we have a model $\M$ of $\varphi$ where $\M(\alpha)$ has $n$
elements, then there is a model $\M' \models \varphi$ where
$\M'(\alpha)$ has $n+1$ elements and for every sort $\beta \neq
\alpha$ we have $\M'(\beta) = \M(\beta)$.
\end{definition}

\subsection{Simple calculus (extend only by copy)}

- example where doesn't work

\subsection{Better calculus (extend by true/false/copy)}

\subsection{ "Biggest" sort, finding injective functions, infinite sorts...}

-Perhaps move to Future work.
