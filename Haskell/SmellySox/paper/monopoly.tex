\section{Monotonicity inference in practice with Monopoly}

We have implemented the monotonicity calculus as part of our tool
Monopoly.

\subsection{NP completeness of Monotonicity}

  A SAT-problem can be encoded as a monotonicity problem as follows:

  Each literal l corresponds to a predicate $p_l$ with one argument.
  The literal being true in the SAT-problem corresponds to the predicate
  being extended by true, and likewise if the literal is false in the SAT-
  problem, the corresponding predicate should be extended by false. 
  
  In each clause of the SAT-problem, at least one literal must be true.
  For each clause $(l_1 \vee ...\vee l_n)$, we create a formula
  $$ \neg p_{l_1}(X) \wedge ... \wedge \neg p_{l_n}(X) \Rightarrow X = c $$
  This formula is monotone in our calculus exactly when the variable $X$ 
  is guarded by at least one of the predicates $p_{l_1},...p_{l_n}$.
  Thus, if we can show that there is a consistent extension of the predicates
  which makes the formula monotone, then the original set of propositional 
  clauses is satisfiable.

\subsection{...}

  In the implementation of Monopoly, we use a SAT-solver to find the context 
  in which to extend the predicates. It works as follows:

  For each predicate $p$ occuring in the sorted problem, we create two literals;
  $p_T$ and $p_F$. If $p_T$ is true, then $p$  is extended by true. 
  If $p_F$ is true, then $p$ is extended by false. Since a predicate can only
  be extended in one way, we add to our SAT formula the clause $\neg p_F \vee \neg p_T$.
  If both $p_T$ and $p_F$ are false, this means that $p$ is extended by copying.


  


