\section{Monotonicity inference in practice with Monopoly}


We have implemented the monotonicity calculus as part of our tool
Monopoly.

Monopoly goes through the input formula once for each sort, checking if
the formula is monotone in that sort. If the answer is yes for some sort $\alpha$,
it is safe to let a quantification over $\alpha$ scope over the entire domain,
and the sort can thus be removed while preserving satisfiability. 

If the answer is no, a sorting predicate or sorting function (depending on the
flag provided by the user) must be introduced in order to make the sorted and the 
unsorted formula equisatisfiable.
 
\subsection{NP completeness of finding a context}

  To show that a formula is monotone in a sort $\alpha$, we must show that there is
  a consistent way of extending the domain of $\alpha$. 

  As explained in \ref{...}, if equality between two terms of type $\alpha$ occurs 
  positively in a clause, this may restrict the size of the domain of $\alpha$. But if 
  the equality is "guarded" by a predicate $p$ in the clause, then the clause is still monotone.

  For a predicate $p$ to be a guard, it must be extended by $false$ if it occurs negatively,
  and with $true$ if it occurs positively.To find if there is a consistent extension of $p$, 
  we must consider the entire formula, since different clauses may require conflicting 
  extensions for $p$, as in example \ref{ex:extension_conflict}.

  The problem of finding such a context in which to extend the model is NP complete, as a SAT-problem can be 
  encoded as a problem of finding a model extension as follows:

  Let each literal $l$ in the SAT-formula correspond to a predicate $p_l$ with one argument.
  The literal being $true$ in the SAT-problem corresponds to the predicate
  being extended by $true$, and likewise, if the literal is $false$ in the SAT-
  problem, the corresponding predicate should be extended by $false$. If the value of $l$ does
  not affect the satisfiability of the SAT-formula, the extension of the predicate does not
  matter either (it is never needed as a guard), so it can simply by extended by copying. 
  %TODO define "extension by copying".

  In the SAT-problem, at least one literal in each clause must be $true$.
  We model this in the model extension problem, by creating for each clause $(l_1 \vee ...\vee l_n)$ a formula
  $$ \neg p_{l_1}(X) \wedge ... \wedge \neg p_{l_n}(X) \Rightarrow X = c $$
  This formula is monotone in our calculus exactly when the variable $X$ 
  is guarded by at least one of the predicates $p_{l_1},...p_{l_n}$.
  Thus, if we can show that there is a consistent extension of the predicates
  which makes the formula monotone, then the original set of propositional 
  clauses is satisfiable.

  %TODO: Define "guardedness" - 

\subsection{Implementing the calculus}

  Monopoly uses a SAT-solver in the implementation of the monotonicity calculus of section \ref{sec_monotonicity}.
  
  The problem of finding if a formula $\formula$ is monotone in a type $\alpha$ is encoded as a SAT-problem,
  $\satformula$ which tries to assign a context in which to extend the predicates while keeping $\formula$ equisatisfiable
  in an extended domain.

  A solution to $\satformula$ directly corresponds to a context.


If there is such a context, then $\formula$ is monotone in the considered sort, so the sort can be removed.

  By \ref{..}, a formula is monotone in a type if all of its clauses are monotone in the type. 
  $\satformula$ is the conjunction of the $\satlit$

  A clause is monotone in a type $\alpha$ if all variables of type $\alpha$ occuring

  In the implementation of Monopoly, we use a SAT-solver (MiniSAT \ref(...)) to find the context 
  in which to extend the predicates. It works as follows:

  Given the input formula $\formula$, we create a SAT formula $\satformula$ where each predicate $p$ occuring
  in the sorted problem corresponds to two literals; $p_T$ and $p_F$. 

  $\satformula$ is constructed in the following way:
  
  for each literal $l_i$ in $\formula$, a SAT formula ${\satlit}_{i}$ is created in the following way:

  

%  $p_T$ being $true$ in $\satformula$ corresponds to the extension of $p$  being $true$.
%  Similarly, if $p_F$ is $true$ in $\satformula$, then the extension for $p$ is $false$.


  for each literal $l_i$ in $\formula$, a SAT formula ${\satlit}_{i}$ is created in the following way:

  $\satformula$ is then the conjunction of the SAT formulas corresponding to its literals.
  
%  Since a predicate can only be extended in one way, we add to our SAT formula a clause stating
%  that $p_T$ and $p_F$ cannot be $true$ at the same time:  $\neg p_F \vee \neg p_T$.
%  If both $p_T$ and $p_F$ are $false$, this means that $p$ is extended by copying.

  
  The SAT formula is the conjunction of SAT clauses and the constraint that no 
  predicates of the given sort may be true extended and false extended at the same time.

 % The SAT clauses correspond to the clauses of the original formula.

  \begin{algorithm}[t]
\caption{Finding the Context}\label{alg:zoom} 
\begin{algorithmic}[1]
\Procedure{SATFormula}{$clauses$, $sort$} 
    \State \textbf{return}  

%   \State $n \gets size(theory) \;\mbox{div}\; 2$  
%   \While {$n \geq 1$}
%   \State partition $theory$ into partitions $p_i$ of size $n$ \Comment{one partition may have to be smaller than $n$}
%   \For {each partition $p_i$}
%      \If {\Call{TryFindFiniteModel}{$theory - p_i$} fails}               \Comment{try Paradox with given time-limit}
%         \State \textbf{return} \Call{Zoom}{$theory - p_i$}  \Comment{recursively find a better subset}
%      \EndIf 
%   \EndFor
%   \State $n \gets n \;\mbox{div}\; 2$                              \Comment{remove smaller parts in next iteration}
%   \EndWhile
%   \State \textbf{return} $theory$                   \Comment{every subtheory is finitely satisfiable}
\EndProcedure

\Procedure{SatClause}{$clause$, $sort$}
\EndProcedure
\end{algorithmic}
\end{algorithm}




  


