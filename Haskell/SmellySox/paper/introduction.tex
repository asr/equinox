\section{Introduction}

Many problems are naturally expressed in sorted logic.

For example, in geometry, the universe can be divided into 
objects that are points, planes, lines etc...

sorted FOL has the same properties as  FOL.

Theorem provers - none exist for sorted logic...
Translate sorted to unsorted! Now we can use any FOL prover.

Sorted logic example:

\begin{example}[werewolves and villagers] As a running example, we
model a village that has been infested with werewolves.\footnote{Our
example is an oversimplification of the party game Werewolf, also
known as Mafia.} The werewolves only attack at night, and during the
day they are ordinary villagers, so no-one knows who they are. In
desperation, the villagers pick some unfortunate people to be burnt at
the stake. The village is saved if this action kills all the
werewolves.

We can model this situation with the aid of two sorts $\villager$ and
$\werewolf$. We have a predicate $\slain$ that characterises which
villagers were burnt alive and a function $\alterEgo$ that says who
each werewolf is during the day:
\label{ex:werewolf1}
\begin{align*}
\slain & \in \villager \to o \\ 
\alterEgo & \in \werewolf \to \villager
\intertext{To make the problem more interesting for Monopoly, we
only allow the villagers to kill at most one person, a \emph{victim}
of their choosing:}
\victim & \in \villager 
\end{align*}
We state that the $\alterEgo$ function must be injective and no-one other
than the victim may be killed:
\begin{align*}
\forall X,Y \in \werewolf \ & ( \alterEgo (X) = \alterEgo (Y) \Longrightarrow  X = Y) \\
\forall X \in \villager\ & (\slain(X) \Longrightarrow X = \victim) \\
\intertext{Finally, we assert that the villagers have won---all
  werewolves were slain:}
 % & \slain(\victim) \\
 \forall X \in \werewolf \ & (\alterEgo(X) = \victim)
\end{align*} 
\end{example}

We want to use a model finder on this example to see if it is possible
for the villagers to win the game. But model finders such as Paradox
only operate on unsorted first-order logic! Luckily, we can use
Monopoly to safely remove all the sorts from our formula and we get:
\begin{align*}
\forall X, Y \ & ( \werewolfp(X) \land \werewolfp(Y) \land \alterEgo(X) = \alterEgo(Y) \Longrightarrow X = Y) \\
\forall X & (\slain(X) \Longrightarrow X = \victim) \\
\forall X \ & (\werewolfp(X) \Longrightarrow \alterEgo(X) = \victim) \\
& \werewolfp(\skolemconstant)
\end{align*}
For soundness Monopoly has introduced a predicate $\werewolfp$ that
characterises the werewolves but has detected that no such predicate
is necessary for villagers. We will see later why this is the case.

What we can do though is to translate sorted logic to unsorted logic. 
(this is possible since sorted logic and unsorted logic are equally expressive) 
We can then use an existing model finder (or theorem prover) for unsorted logic.

How to translate?

To simply remove the sorts is unsound, as we shall see in section ...
Introduce sorting predicates - inefficient.

Use monotonicity to infer when it is necessary to introduce sorting predicates
and when it is safe to remove the sorts. (Mention Jasmins paper)

- were going to show you lots of cool stuff in the next sections. blahblah.



%TODO cite Enderton 1972 - translation sorted - unsorted by predicates.










