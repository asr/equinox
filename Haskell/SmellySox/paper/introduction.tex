\section{Introduction}

Many problems are naturally expressed in sorted logic.

For example, in geometry, the universe can be divided into 
objects that are points, planes, lines etc...

sorted FOL has the same properties as  FOL.

Theorem provers - none exist for sorted logic...
Translate sorted to unsorted! Now we can use any FOL prover.

Sorted logic example:

Explain basics of werewolf game (also known as "mafia").



\begin{example}
\label{ex:werewolf1}
\begin{eqnarray*}
\slain & \in & \villager \to o \\ 
\alterEgo & \in & \werewolf \to \villager \\
\victim & \in & \villager 
\end{eqnarray*}
\begin{align*}
\forall X,Y \in \werewolf \ & ( \alterEgo (X) = \alterEgo (Y) \Longrightarrow  X = Y) \\
 \forall X \in \villager\ & (\slain(X) \Longrightarrow X = \victim) \\
 & \slain(\victim) \\
 \forall X \in \werewolf \  & (\alterEgo(X) = \victim)
\end{align*} 
\end{example}

We want to use a model finder on this example to see if it is possible
for the villagers to win the game. But there are no model-finders for
sorted first-order logic!

What we can do though is to translate sorted logic to unsorted logic. 
(this is possible since sorted logic and unsorted logic are equally expressive) 
We can then use an existing model finder (or theorem prover) for unsorted logic.

How to translate?

To simply remove the sorts is unsound, as we shall see in section ...
Introduce sorting predicates - inefficient.

Use monotonicity to infer when it is necessary to introduce sorting predicates
and when it is safe to remove the sorts. (Mention Jasmins paper)

- were going to show you lots of cool stuff in the next sections. blahblah.



%TODO cite Enderton 1972 - translation sorted - unsorted by predicates.










